\documentclass[a4paper, 12pt]{article}

\usepackage[english]{babel}
\usepackage[utf8]{inputenc}
\usepackage[T1]{fontenc}
\usepackage[a4paper]{geometry}
\usepackage{amsmath}
\usepackage{amssymb}
\usepackage{indentfirst}
\usepackage{graphicx}
\usepackage{hyperref}
\usepackage{float}
\usepackage{csquotes}
\renewcommand{\rmdefault}{ptm}

\title{How social debt in software development affects your motivation in the graduation}
\author{Leonardo Lana Violin Oliveira, \\ oriented by: Alfredo Goldman and \\ Damian A. Tamburri }

\begin{document}
\maketitle

\section*{Introduction}
The research has as its goal to verify if technical debts, that rise as progress towards the end of
the assignment generates a social debt. Students not always implement their programs
in an organized way or in a way that let the code's reading easy, which can generate a
technical debt. With that in mind, I intend to study if these situations have an effect
lowering students' enjoyment, well-being and  attention for the assignments and even the
course itself, therefore getting into a sub-optimal routine when developing the
next assignments.

\section*{Methodology}
The data gathering will be done through Github's private repositories, online forms
and register interviews with the students, who are attending to MAC0216
(\textbf{Programming Techniques I}) and agree with their data being collected.

The interviews will be made with each group in the following weeks after the deadline of
assignment, and the Github's repositories will be checked weekly.

What motivated the choice of MAC0216 for the project was the fact that it will be the first
time that first-year students will have contact with group assignments, therefore it is 
an environment prone to technical debts (bad modularized functions and files, etc.).
Furthermore, I'm the teacher's assistant of this subject, so if necessary I will provide
a \textit{git} course to the students.

\section*{Form}
The form in question is hosted at: \url{goo.gl/forms/xJYaQqsyvahguv8J3}

The form was divided into three parts: identification of the student (name and
USP identification number), creating the student's profile (assertive questions)
and final questions (discursive questions).

\subsection*{Assertive questions}
This questions have the objective of mounting the student's profile, 
they ask:
\begin{enumerate} 
\item Previous level of experience of programming
\item How organized is the code
\item If the code has commentaries
\item If there is a plan before programming
\item How much time was spent in refactoring
\item In which time frame the assignment is completed
\item How the assignment is developed
\end{enumerate}

The goal of the first question is to better understand how the students perceive their experience, 
so we can take caution when inferring which basic programming techniques they know (commentaries
structure, modularization, version control, etc.). 

Questions 2, 4, 6 and 7 seek to understand how organized the student is, for example, if the student 
completes the assignment near the deadline and develop it in peaks, these answers may suggest that 
he/she is organized in the coding aspect as well as scheduling aspect.

Questions 2, 3, and 5 seek to understand how the student codes, if him/her is organized within the
code's realm, he/she documents the code properly with commentaries and if in the sight of design problems, 
he/she applies refactoring.

\section*{Updates}
I will post every other week an update, with the current data analysis and conclusions
in a form of a \textbf{.pdf} file in this website: 
\url{leolanavo.github.io/MAC0215/}

\nocite{*}
\bibliographystyle{IEEEannot}
\bibliography{annot}

\end{document}
